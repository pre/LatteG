\documentclass{tktltiki}
\usepackage{ae,aecompl}
\usepackage{url}
\usepackage{amsfonts}
\usepackage{color}
\usepackage{graphicx}

% rubber: module pdflatex
% rubber: pdflatex.options -Ppdf -t a4

\begin{document}

\title{Salasanoista ei päästä eroon.\\ --- Mutta äheltämisestä päästään!}
\author{Petrus Repo}
\date{\today}
\level{Sosiaalisen median tekniikat -seminaari}
\maketitle


\onehalfspacing

\level{Seminaarityö}
\faculty{Matemaattis-luonnontieteellinen}
\department{Tietojenkäsittelytieteen laitos}
\subject{Tietojenkäsittelytiede}
\numberofpagesinformation{\numberofpages\ sivua}

\keywords{}

\begin{abstract}

\end{abstract}

\setcounter{tocdepth}{3}
\mytableofcontents


\section{Johdanto}

Jesjes.

\section{Autentikoituminen ja web}
\subsection{Autentikointi on identiteetin määrittämistä}
\subsection{Challenge--response-autentikointi}
\subsection{Kaksiosainen autentikointi}
\subsection{Client-sertifikaatit}

\section{Menetelmiä autentikoitumiseen webissä}
\subsection{Palvelukohtainen oma salasana}
\subsection{OpenID}
\subsection{OAuth}
\subsection{WebID}

\section{OpenID autentikointistandardina}
\subsection{Periaate ja tavoite}
\subsection{Hyviä puolia}
\subsection{Huonoja puolia}
\subsection{Näkemys OpenID:stä}

\section{Salasanoista ei päästä eroon}
\subsection{Katsaus olemassaoleviin toteutuksiin}
\subsection{Vaatimukset pilviyhteensopivalle autentikoinnille}

\section{Jatkotutkimuksia}
\section{Yhteenveto}


\bibliographystyle{tktl}
\bibliography{lahteet}

\lastpage

\end{document}